% ============================================================================
% CHAPTER 6: CONCLUSION
% ============================================================================

\chapter{CONCLUSION}


This research developed a Multi-Modal Hybrid Neural Network framework for Android application privacy risk classification, combining Variational Autoencoder-based structural feature learning with DistilBERT semantic embeddings through attention-based fusion. The model achieved 77.48\% accuracy and 0.8901 macro-AUC, outperforming single-modality baselines and five state-of-the-art methods. Ablation studies identified focal loss and attention mechanisms as critical components for handling class imbalance and enabling effective multi-modal integration. The framework was trained on curated Nepali-region Android applications and validated through rigorous cross-fold evaluation with statistical significance testing.

All research objectives were accomplished: the multi-modal architecture was designed and empirically validated; class imbalance was mitigated through specialized loss functions achieving balanced per-class performance; the framework was evaluated on real-world data with comprehensive metrics; systematic benchmarking confirmed superior performance over existing approaches; and computational feasibility for deployment was demonstrated. This work establishes that task-specific architectural design with intelligent fusion mechanisms enables effective automated privacy risk assessment for mobile applications, providing both methodological contributions and practical foundations for deployment in app distribution ecosystems.

% ============================================================================
% END OF CONCLUSION CHAPTER
% ============================================================================