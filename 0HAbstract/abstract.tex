\chapter*{\centerline{ABSTRACT}}
\addcontentsline{toc}{chapter}{ABSTRACT}
\thispagestyle{plain}
\vspace{-0.5cm}

Mobile applications often request access to device resources that go beyond their actual functional needs, creating substantial privacy concerns for users. While existing security frameworks focus mainly on malware detection, they offer limited support for evaluating the privacy sensitivity or contextual justification of requested permissions. To address this gap, this research presents a \textbf{Multi-Modal Hybrid Neural Network (MM-HNN)} framework for the supervised classification of Android application privacy risk using app metadata, manifest-declared permissions, genre information, and textual descriptions.

The proposed framework integrates two complementary learning components. A \textbf{Variational Autoencoder (VAE)} is used to learn compact latent representations from permission and metadata vectors and is augmented with a \textbf{classification layer} trained on manually assigned privacy risk labels. In parallel, \textbf{DistilBERT} is employed to extract contextual semantic features that reflect the alignment between requested permissions and the declared functionality of applications. These latent structural features and semantic embeddings are then fused to construct a multimodal classifier capable of identifying potentially privacy-invasive behavior.

The MM-HNN model is trained and evaluated using a labeled dataset of Android applications collected from the Nepali region. Standard performance metrics, including accuracy, F1-score, AUC, and confusion matrices, are used for evaluation. The results indicate that the multimodal fusion approach consistently outperforms single-modality baselines by jointly capturing structural irregularities and contextual relevance. Operating within Android’s permission framework and Google Play Store guidelines, the proposed system offers a practical and data-driven foundation for automated mobile privacy risk assessment.
\par
\textbf{Keywords:} \textit{Multi-Modal Neural Networks, Variational Autoencoders, Privacy Risk Classification, Permission Analysis, Semantic Alignment, Android Privacy}